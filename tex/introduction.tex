
It is widely known that statistical data editing (data cleaning) is an
essential step in the production of accurate and reliable official statistics.
Indeed, the impact on estimated values of data editing and imputation
procedures has beem widely researched. \citet{whitridge2006impact} for example,
show that imputation of missing values can have a significant effect on
estimated totals of economic variables and \cite{loo2014towards} show that
estimated means and naively estimated variances may vary significantly over a
multi-step data editing process. \cite{dasu2012statistical} propose metrics to
evaluate the impact of data cleaning based on distributional properties pre-
and post data cleaning in the context of very large databases.\todo{Jeroen: heb jij
hier nog wat extra referenties?}

Besides the changes of estimated means, variances, and other distributional
parameters in edited data, the precision of these estimates is influenced by
data cleaning as well. In specific, the case of variance caused by imputation
has received considerable attention over the last decades. The methods based on
bayesian bootstrapping by \cite{rubin1987multiple,rubin1996multiple} and the
resampling based methods of \cite{rao1992jackknife,rao1996variance}, and
\citet{shao1996bootstrap}  \cite{shao2002sample,shao2008confidence} are amongst
the most well-known. However, a statistical data editing process usually
consists of a sequence of steps where estimation of missing values is just one
of the steps performed.  Other steps, such as outlier detection, rule-based
corrections, or error localization may well contain stochastic substeps or
parameter estimations based on observed data that influence the variance of
estimation.

In this paper, the variance caused by data cleaning is therefore studied from a
general point of view, abstracting away from the specific type of data cleaning
step performed. In the next Section it is shown how data cleaning can be
considered a part of the estimator of a parameter and details are provided on
how data editing steps can contribute to the variance of estimation. A general
bootstrap procedure is outlined that can be used to estimate the contribution
of these steps.  Section~\ref{sect:data} provides the details of a simulation
study on a economic (business survey) data set with a realistic set of editing
constraints and sequence of methodologies. The results are discussed in
Section~\ref{sect:discussion} and Section~\ref{sect:conclusion} presents
general conclusions.












